\def\DB{ObsLocTAP database }

\section{Introduction}
We are required to publish the observing schedule and choose ObsLocTap\footnote{\url{https://www.ivoa.net/documents/ObsLocTAP/}} as the protocol.

\section{Requirements}

\citeds{LSE-30} (OSS) requires publication at least two hours ahead of observing.

In \citeds{DMTN-199} the agencies insist we publish a schedule 24 hours ahead of observing.
The lack of fidelity of such a schedule is acknowledged by the agencies.



\section{Design}


There are two parts to this:


\begin{enumerate}
\item The forecast  which comes from the scheduler.
\item The history of which observations were made which we will need to construct.
\end{enumerate}

\subsection{Scheduler forecast}
The scheduler puts out an event record with a number of observations in it\footnote{\url{https://ts-scheduler.lsst.io/developer-guide/developer-guide.html#operation-modes}}.
Since we are required to give 24hours notice this record should always cover 24 hours but that is not
yet in the requirements.
Currently the requirement is 2 hours.
But let us assume the record will be for 24 hours.\footnote{
At 24 hours before we would, at the start of the observing night, need to produce the schedule  \emph{for the next night}.
}
This record may be produced as frequently as every observation is made with an updated schedule depending on weather etc.
Basically when the scheduler has an update a new event is published to SAL.

These events are available at the USDF in the EFD and the last may be found with this influx query:
\begin{lstlisting}
SELECT * FROM "efd"."autogen"."lsst.sal.Scheduler.logevent_predictedSchedule
            where numberOfTargets > 0 ORDER BY DESC LIMIT 1
\end{lstlisting}

The current structure of this record is a bunch of columnar arrays representing a time and pointing with the
values : {\tt mjd, ra, dec, rotSkyPos, nexp}.
It is understood more values may follow such as field/tile and filter.


{\bftext It seems a separate 24 hour schedule produced each night is preferable to the scheduler team. This makes the forecast easy. There would then also be several shorter {\textt predictedSchedule} events to deal with. }

\subsubsection{Handling the schedule}
We will need to look at each schedule event record in the EFD as it arrives.
For our purposes either polling each 30s or having an influx callback on the EFD at USDF would
work.

An initial prototype could simply publish this to a web page or JSON file.
This may be a useful tool in any case.
However for ObsLocTap a separate record must be created for each pointing in a database lets call this the \DB in this document.
Putting this in a postgress database would seem feasible and allow a fairly standard TAP implementation to be deployed atop.
Postgress is already available at USDF.
The schema shall be  defined in the schema repository
IVOA ObsLocTap considers a table called {\texttt obsplan} and the schema is fixed in the spec.
\footnote{ SDM Schema repo \url{https://github.com/lsst/sdm_schemas/tree/main/yml}}
 allowing us to generate the SQL from the description in the repo.

So let us assume we shall have a postgress table with the IVOA fields.
As each new event record is seen the new pointings not already in the database must be added.
If there is a 24 hour schedule some of the new pointings will replace existing ones, the existing entries should be marked {\texttt unscheduled}.
The match would have to be done on proximity in time.

\subsection{History of observations}
Next we need to record which observations were made to fill the {\textt ObsDataSet}.
There is no ID in the scheduler forecast so there is no explicit matching of observation to forecast.
For every observation made we shall have to do either:
\begin{enumerate}
\item If there is a record in the \DB for this pointing mark it as done. This implies further fields are needed to record the observation {\tt Observed:bool, timeObjs:mjd , exposure:long }. We may want to duplicate more than the time to this table but the exposure ID allows one to get any info we needed from the butler registry.
\item It there is no record create a record with {\tt Observed=False} and all the forecast field blank.
\end{enumerate}

This table would now overlap the exposure log but it is not obvious we should combine these as they serve slightly different purposes.
It is further noted by the scheduler team that very few of the scheduled observations would be made as planned. The

\subsubsection{Was the  observation made ?}\label{sec:made}
There is probably no clear cut answer to this given that there is no ID associated with the forecast.
If we assume users of ObsLocTap cared about co-observing then their object could be anywhere in the
$9 deg^2$ area around the pointing.
For a first stab lets say the observation was made if all the following are true :

\begin{enumerate}
\item The filter used matched the filter in the forecast.
This would seem to fairly important but perhaps it is not - I am unsure of the probability of
doing the scheduled observation with the wrong filter. It seems unlikely.
\item The {\tt ra, dec} of the forecast falls within $4 deg$ of the observation made.
This give a 50\% chance or so that any point an observer was interested in was actually in the FOV.
\item The shutter close time of the observation was within 30s of the mjd of the forecast.
We are unlikely to be exactly on time so one exposure error seems reasonable.
\end{enumerate}

This is somewhat arbitrary but at least provides a codeable starting point for discussion.

\subsubsection{Finding the observations}
The butler at the USDF is ingesting the images from the summit within a minute or so.
The butler registry can then provide the metadata needed for  \secref{sec:made}
Again whether this is polling or call back should be discussed.
The prompt processing is trigger at USDF would provide the correct event for doing this for example
since it fires on each new image.
