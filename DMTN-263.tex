\documentclass[DM,authoryear,toc]{lsstdoc}
\input{meta}

% Package imports go here.

% Local commands go here.

%If you want glossaries
%\input{aglossary.tex}
%\makeglossaries

\title{ObsLocTap: Publishing the Rubin Observing Schedule}

% Optional subtitle
% \setDocSubtitle{A subtitle}

\author{%
William O'Mullane
}

\setDocRef{DMTN-263}
\setDocUpstreamLocation{\url{https://github.com/lsst-dm/dmtn-263}}

\date{\vcsDate}

% Optional: name of the document's curator
% \setDocCurator{The Curator of this Document}

\setDocAbstract{%
We are required to publish the Rubin observing schedule some hours in advance.  Following our VO fist strategy  IVOA ObsLocTap is a suitable protocol for doing this. The required  information is available from the scheduler and already in the EFD.
}

% Change history defined here.
% Order: oldest first.
% Fields: VERSION, DATE, DESCRIPTION, OWNER NAME.
% See LPM-51 for version number policy.
\setDocChangeRecord{%
  \addtohist{1}{YYYY-MM-DD}{Unreleased.}{William O'Mullane}
}


\begin{document}

% Create the title page.
\maketitle
% Frequently for a technote we do not want a title page  uncomment this to remove the title page and changelog.
% use \mkshorttitle to remove the extra pages

% ADD CONTENT HERE
% You can also use the \input command to include several content files.

\appendix
% Include all the relevant bib files.
% https://lsst-texmf.lsst.io/lsstdoc.html#bibliographies
\section{References} \label{sec:bib}
\renewcommand{\refname}{} % Suppress default Bibliography section
\bibliography{local,lsst,lsst-dm,refs_ads,refs,books}

% Make sure lsst-texmf/bin/generateAcronyms.py is in your path
\section{Acronyms} \label{sec:acronyms}
\addtocounter{table}{-1}
\begin{longtable}{p{0.145\textwidth}p{0.8\textwidth}}\hline
\textbf{Acronym} & \textbf{Description}  \\\hline

DB & DataBase \\\hline
DESC & Dark Energy Science Collaboration \\\hline
DM & Data Management \\\hline
DMTN & DM Technical Note \\\hline
EFD & Engineering and Facility Database \\\hline
FOV & field of view \\\hline
IVOA & International Virtual-Observatory Alliance \\\hline
JSON & JavaScript Object Notation \\\hline
LSE & LSST Systems Engineering (Document Handle) \\\hline
OSS & Observatory System Specifications; LSE-30 \\\hline
SAL & Service Abstraction Layer \\\hline
SQL & Structured Query Language \\\hline
TAP & Table Access Protocol \\\hline
USDF & United States Data Facility \\\hline
VO & Virtual Observatory \\\hline
deg & degree; unit of angle \\\hline
\end{longtable}

% If you want glossary uncomment below -- comment out the two lines above
%\printglossaries





\end{document}
